\chapter*{Conclusion}
\addcontentsline{toc}{chapter}{Conclusion}

The goal was to build a robust library that can be a basis for 3D editors. The robustness
is based on the template-like implementation technique that allow us to let it work over
any implementation of the mesh. The secondary goal was to establish a convention
for the future expansion of the library that has to be simple and clear. As the
concepts philosophy is clear, the library is now easy to expand with a set of algorithm.\\
\\
Since most of meshes used nowadays are used in the rendering software, the implementations
are commonly not appropriate for algorithms that affects the topology. That is the main reason
why the implementations of algorithms has own implementation of the mesh that loads the
data from an external file or has own convertors. The Hmira library provides a 
solution that saves a user from developing conversion software between the implementations of the mesh.
In addition, it allows a user to create an adaptor for the specific operations
rather than implement a convertor and the implementation of the operation separatedly.\\
\\
The attached examples shows that the metaprogramming technique used in the library brings a great
benefits that allow us to create any algorithm as generic. The challenging part of the implementation
of the algorithm is an algorithm decomposition; as the algorithm is decomposed, the algorithm
is then ready to implement using elementary operations that are contained in the concept.\\
\\
As the project is developed on public repository \texttt{https://github.com/hmira/Medved},
anyone is allowed to expand the library using the instructions contained in the appendicies.
One can implement algorithms, adaptors for the operations and the traits for the
mesh library.