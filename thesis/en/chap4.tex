\chapter{Designing the Library}

There are a lot of libraries that offer the ability to represent the volume data
and the basic manipulation with it. There are a lot of published algorithms
that can be implemented as well.\\
\\
Our goal is to create a generic set of algorithms that can be used over any implementation
of mesh that satisfies a required concept. Before the algorithm is finally implemented,
we must completely describe the concept of the algorithm and the behavior of the algorithm
on a closely unspecified mesh. The point of this goal is \emph{to think generally} regardless of
the specification of a potentionally used mesh.\\

\section{Observation}

Observing the algorithms step-by-step, we can see that single steps
of the algorithm are the variations of adding, removing and modifying the elements
of the mesh. In addition, the algorithms also uses a capabilities of querying
in mesh such as getting all adjancent vertices of given vertex or getting all
vertices in a container or any iterable structure.\\
\\
Such capabilities of the mesh are critical for implementing an algorithm.
The question then arises, ``Why do we implement the algorithms for meshes if
the usage of required operations might suffice?" The answer can be relatively simple:
We are generally not able to determine how is a given structure controlled
without being provided the an additional information.\\
\\
Thus there is an option to build a correctly working algorithm over any structure
that fullfills the criteria that has been presented by the algorithm. Nowadays in C++,
we are able to build a template-based interface that modifies a given structure according
to algorithm requirements during a compile time.\\
\\
Before using this set of algorithms, the user only defines an interface for chosen structure
and passes it as a parameter.

\section{Concepts and Algorithms}

The concept is specific according to the requirements of an algorithm. Let us provide an example
of an algorithm and then build a structure concept that requires the algorithm in order to work 
correctly.\\

\textbf{Example:}\\
The algorithm that flips normal on each face of a given mesh.\\

\textbf{Solution:}\\
Before analyzing the algorithm, we can see that input is a mesh (not closely specified)
that returns modified mesh. The idea of the algorithm is simple; pick all faces
and flip the normal of each one. Obviously, the mesh is required to have faces. This is not an
unnecessary note, because there are several implementations of mesh without faces; e.g.
vertex-edge mesh. Finally, let us summarize what \emph{attributes} are required by the algorithm.

\begin{itemize}
\item faces of a mesh
\end{itemize}

Now, when we have the required attributes, we need to specify which \emph{operations} are required.

\begin{itemize}
\item flipping the normal of a face 
\item capabilty of applying the operation on each face
\end{itemize}

\textbf{Question:}\\
What if the structure does not have a method that flips the normal of a face?\\
\\
After all, the algorithm can be build without providing a method that flips a normal;
it can flip the normal of a face by itself.
There are two possible approaches to solving this question. The first: create the
next algorithm with different requirements separately. The second: build an \emph{adapter} that
has the same functionality as a required method that flips normals and use it as method
in the algorithm described above instead a required method.\\
\\
Both cases require additional points of the concept instead the ommited one.
In order to reverse the normal of a face manually, the following is required:

\begin{itemize}
\item the normal of a face
\item getting the face normal
\item setting the face normal
\end{itemize}
It is up to us how do we design the algorithms and the level of the implementation. The more sophisticated
the adapters are, the bigger the number of cases of mesh implementation can be covered. If the compiler does not
find the required methods and it is able to build an adapter, then it does not bother the user and it builds
a modified method. Naturally, adapter may have own requirements therefore we can assume that
some point of concepts can be replaced with different ones.