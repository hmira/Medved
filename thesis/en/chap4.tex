\chapter{Goals of the thesis}

There are a lot of libraries that offers the ability to represent the volume data
and the basic manipulation with it. There are also lot of published algorithms
that can be implemented.\\

Our goal is to create a generic set of algorithms that can be used on any implementation
of mesh that satisfies required concept. Before the algorithm is finally implemented,
we must completely describe the concept of the algorithm nor the behavior of algorithm
on specified mesh. The point of this goal is \emph{to think generally} regardless of
the specification of a potentionally used mesh.\\

\section{Observation}

Observing the algorithms step-by-step we can see that a single steps
of the algorithm are the variations of adding, removing and modifying the elements
of the mesh. In addition, the algorithms uses also a capabilities of querying
in mesh such as getting all adjancent vertices of given vertex or getting all
vertices in a container or any iterable structure.\\

Such a capabilities of the mesh are critical for implementing an algorithm.
The question then arises, ``Why do we implement the algorithms for meshes if
the usage of required operations might have suffice?" The answer can be relatively simple:
We are generally not able to determine how is a given structure controlled
without giving additional information.\\

Thus there is an option to build an algorithm working correctly over any structure
that fullfills the criteria that has been presented by the algorithm. Nowadays in C++,
we are able to build a template-based interface that modify a given structure according
to algorithm requirements during a compile time.\\

Before using this set of algorithms the user only defines a interface for chosen structure
and pass it as parameter.

\section{Concepts and algorithms}

Concept is specific according the requirements of an algorithm. Let us provide an example
algorithm and then we build a structure concept that requires the algorithm in order to work 
correctly.\\

\textbf{Example:}\\
The algorithm, that flips normal on each face of a given mesh.\\

\textbf{Solution:}\\
Before analyzing the algorithm, we can see that input is a mesh (not closely specified)
that returns modified mesh. The idea of the algorithm is simple; iterate through all faces
and flip the normal of each one. Obviously, the mesh is required to have a faces. This is not an
unnecessary point because there are several implementations of mesh without faces; e.g.
vertex-edge mesh. Finally, let us summarize what \emph{attributes} are required by the algorithm.

\begin{itemize}
\item faces of a mesh
%\item normal of a face
\end{itemize}

Now when we have the required attributes, we need to specify which \emph{operations} are required.

\begin{itemize}
\item iterating through all faces of mesh
\item flipping the normal of a face 
\end{itemize}

\textbf{Question:}\\
What if the structure does not have a method that flips the normal of a face?\\

After all, the algorithm can be build without providing a method that flips a normal;
it can flip the normal of a face by itself.
There are two possible approaches of solving this question. The first: create separately
next algorithm with different requirements. The second: build an \emph{adaptor} that
has the same functionality as required method that flips normals and use it as method
in the algorithm described above instead required method.\\

Both cases requires additional points of concept instead the ommited one.
In order to reverse the normal of a face manually, it is required:

\begin{itemize}
\item the normal of a face
\item getting the face normal
\item setting the face normal
\end{itemize}

It is up to us how we design the algorithms and the level of implementation. As more sophisticated
the adaptors are than more cases of mesh implementation can be covered. If the compiler does not
find the require methods and is able to build an adaptor then does not bother the user and build
a modified method. Naturally, adaptor may have own requirements therefore we can assume that
some point of concepts can be replaced with different ones.