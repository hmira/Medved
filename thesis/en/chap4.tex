\chapter{Goals of the thesis}

There are a lot of libraries that offers the ability to represent the volume data
and the basic manipulation with it. There are also lot of published algorithms
that can be implemented.\\

Our goal is to create a generic set of algorithms that can be used on any implementation
of mesh that satisfies required concept. Before the algorithm is finally implemented,
we must completely describe the concept of the algorithm nor the behavior of algorithm
on specified mesh. The point of this goal is \emph{to think generally} regardless of
the specification of a potentionally used mesh.\\

\section{Observation}

Observing the algorithms step-by-step we can see that a single steps
of the algorithm are the variations of adding, removing and modifying the elements
of the mesh. In addition, the algorithms uses also a capabilities of querying
in mesh such as getting all adjancent vertices of given vertex or getting all
vertices in a container or any iterable structure.\\

Such a capabilities of the mesh are critical for implementing an algorithm.
The question then arises, "Why do we implement the algorithms for meshes if
the usage of required operations might have suffice?``

\section{Concepts of an algorithms}

Concept is specific according the requirements of an algorithm.