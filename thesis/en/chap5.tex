\chapter{Algorithms Decomposition}

In this chapter are decomposed the algorithms (see chapter \ref{chap:op_al}) so that basic
operations over polygon
mesh can be summarized. The word ``\emph{basic}" is relative according the chosen level.
In this thesis is chosen such level that leads to the least count of operations in total.\\
\\
If the mesh does not fulfill the concept of the algorithm \emph{adapters} can be used instead
of the absent operators. An adapter behaves like an operator but in fact it is the algorithm that
makes a desired step as a result. In the other words it is the operator implemented in terms
brute-force. In some cases, lack of an operator can not be replaced by an adapter.\\
\\
The algorithms described in this chapter are only the small part of all algorithms. The emphasized
algorithms are primarily the commonly used, the algorithms that changes the mesh topology and
the conversion algorithm between voxel and polygon mesh representation.

\section{Converting between representations}

\subsection{Marching cubes}

The algorithm has the specific requirements for the mesh and also for the voxel map (See algorithm
description \ref{sub:march}). However the voxel map can be in various forms some special cases
of marching cubes are explained.

\algblockdefx[ParallelFor]{ParallelFor}{EndParallelFor}
[1]{\textbf{parallel for} #1}
{\textbf{end parallel for}}

\begin{algorithm}[H]
\caption{Marching cubes}
\label{alg:mc}

\begin{algorithmic}[1]
	\Function{Marching cubes}{$voxelMap$}
			\State $mesh$ := empty mesh

		\ParallelFor{\textbf{each} $voxel$ \textbf{in} $voxelMap$}
			\State $cubeType$ := Determine Cube Type($voxel$)
			\State $faces$ := Create Faces($cubeType$)
			\For{\textbf{each} $face$ \textbf{in} $faces$}
				\State Add Face To Mesh($mesh$, $face$)
			\EndFor
		\EndParallelFor
		\State \Return{$mesh$}
	\EndFunction
	\\
	\Function{Determine cube type}{$voxel$}
		\State $configuration$ := initial configuration
		\State \Comment{configuration contains all the information}
		\State \Comment{	about each corner of the cube}
		\State \Comment{initial configuration assumes that all corners are outside}
		\State $corners$ := Get Voxel Corners($voxel$)
		\For{\textbf{each} $corner$ \textbf{in} $corners$}
			\If {$corner$ is inside the surface}
				\State Set in configuration the $corner$ as inside
			\EndIf
		\EndFor
		\State \Return{$configuration$}
	\EndFunction
\end{algorithmic}
\end{algorithm}

%\caption{The simplified pseudo-code of the algorithm Marching cubes}


From the pseudocode \ref{alg:mc} we can conclude that the structures used as a parameters
are required to support operations used in the algorithm. Thus the concept of the algorithm
is designed as follows:

\begin{itemize}
\item $mesh$ is required to support $face$ structure \emph{(line 6)}
\item $mesh$ is required to support adding the $faces$ \emph{(line 7)}
\item $voxelMap$ is required to support $voxel$ structure \emph{(line 3)}
\item $voxelMap$ is required to support getting all the $voxels$ \emph{(line 3)}
\item $voxelMap$ is required to support getting all the $corners$ of each $voxel$ \emph{(line 18)}
\item $voxelMap$ and $mesh$ are required to have a $cubeType$ convertible to a set of $faces$ \emph{(line 5)}
\end{itemize}

\subsection{Voxelization}

Compared to the \emph{Marching cubes} algorithm, the requirements for the voxel map structure
are more comprehensive. As mentioned in description \ref{sub:vox}, the algorithm consequently rasterizes
the edges, the faces and finally fills the object.

\begin{algorithm}[H]
\caption{Voxelization}
\label{alg:vox}
\begin{algorithmic}[1]
	\Function{Voxelize mesh}{$mesh$}
			\State $voxelMap$ := map of empty voxels

		\ParallelFor{\textbf{each} $face$ \textbf{in} $mesh$}
			\State Voxelize face($face$, $voxelMap$)
		\EndParallelFor
		\State Run Floodfill($voxel map$)
		\State \Return{$voxel map$}
	\EndFunction
	\\
	\Function{Voxelize face}{$face$, $voxelMap$}
		\For{\textbf{each} $edge$ \textbf{in} $face$}
			\State Voxelize edge($edge$, $voxelMap$)
		\EndFor
		\State Run Face Floodfill($face$, $voxel map$)
		\State{\textbf{update} $voxelMap$}
	\EndFunction
	\\
	\Function{Voxelize edge}{$edge$, $voxelMap$}
		\State $v_{0}$ = First Vertex($edge$)
		\State $v_{1}$ = Second Vertex($edge$)
		\For{\textbf{each} $voxel$ \textbf{between} $v_{0}$ \textbf{and} $v_{1}$}
			\State Set As Filled($voxel$)
		\EndFor
		\State{\textbf{update} $voxelMap$}
	\EndFunction
\end{algorithmic}
\end{algorithm}