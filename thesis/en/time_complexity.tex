\chapter{Computational complexity}

In this chapter is described the impact on the computational complexity by choice of the mesh
representation. We will also define the limits for those we consider the operation
as \emph{effective} or \emph{ineffective}. Later in the thesis, if the operation is
marked as effective it must fulfill the defined criteria. If the mesh representation
does not contain an operation that satisfies the criteria for complexity then
the representation does not natively support the operation.

\section{Efficient vs. Inefficient}

The efficiency is a relative expression so that we need to define what is effective.
If the operation does query on the local topology only then the scanning the whole structure
are considered in this case as ineffective. The acceptable complexity is the complexity
of scanning the immediate surroundings.\\

\textbf{Example:}\\
Getting all adjacent verticies of a vertex in the face-vertex representation. \ref{sec:face-vertex}\\

\textbf{Analysis:}\\
However the structure does not possess information about vertex adjacency
we can acquire the demanded information by scanning the topology of the mesh.
As we do not have any auxiliary structure the information can be acquired only by
scanning the whole structure. Thus the only way how to solve this problem
is for each face check whether the vertex is contained in the face. Then the
previous and the next vertex from the view of the demanded vertex are the ones
that forms the edges in the mesh.\\

\textbf{Conclusion:}\\
The computation complexity of the operation can be computed intuitively:
For \emph{each face} we need to check whether the demanded vertex is contained in the
face so we need to check \emph{each vertex in the face} and if the vertex is contained
we are able get an adjacent vertex in constant time.\\
\\
Thus the computational complexity of the operation
is $\mathcal{O}(F_{mesh} \times V_{face})$ in total.
However getting the adjacent vertices is the operation querying the surroundings only
the complexity of the operation on the face-vertex structure is above the effective limit.\\
\\
The limits for the operation are defined in the following paragraph.

\section{Limits}

For each operation mentioned in \ref{chap:op_al} has to be defined the specific limit
of effectiveness. Before we specify the limits, we classify the operations in two categories:

\begin{itemize}
\item \emph{Global operation} - the operation that can not be done without any information
about entire structure
\item \emph{Local operation} - the operation that involves only the specific part of the mesh
with no demans on the rest of elements in the structure
\end{itemize}
In the following analysis are involved operations that adds/removes an element(vertex, edge, face),
Euler operators\ref{sec:euler}, advanced editing functions\ref{sec:edit_f},
the getters on the surroundings of an element and the getters
on the entire structure(e.g. getting all vertices).

\begin{figure}[!hbf]

\centering
\textbf{Add/remove primitives}\\
\label{fig:comp_prim}
\vspace{2mm}
\begin{tabular}{| c | c | c | c |}
\hline
\textbf{Name of the operation} & \textbf{Allowed complexity} & \textbf{local/global}\\
\hline
Add vertex & $\mathcal{O}(V_{1})$ & \emph{local}\\
\hline
Remove vertex & $\mathcal{O}(V_{adjacentV})$ & \emph{local}\\
\hline
Create edge from vertices& $\mathcal{O}(V_{adjacentV})$ & \emph{local}\\
\hline
Remove edge & $\mathcal{O}(V_{adjacentV})$ & \emph{local}\\
\hline
Create face from vertices& $\mathcal{O}(V_{adjacentV})$ & \emph{local}\\
\hline
Remove face & $\mathcal{O}(V_{adjacentV})$ & \emph{local}\\
\hline
\end{tabular}
\end{figure}
In the table above are shown basic operations over mesh structure. The allowed
computational complexity is based on the number of adjacent vertices of the vertex involved
in the operation. In case of the operation that involves more vertices the number $V_{adjacentV}$
represents the sum of all adjacent verticies of all verticies involved in the operation.
E.g. in case of the operation \emph{Create face from vertices} if the operation creates a
face from the triplet of vertices the computational complexity is $3 \times V_{adjacentV}$.

\begin{figure}[!hbf]

\centering
\textbf{Euler operators}\\
\vspace{2mm}
\begin{tabular}{| c | c | c | c |}
\hline
\textbf{Name of the operation} & \textbf{Allowed complexity} & \textbf{local/global}\\
\hline
MakeV & $\mathcal{O}(V_{1})$ & \emph{local}\\
\hline
KillV & $\mathcal{O}(V_{adjacentV})$ & \emph{local}\\
\hline
MakeEV & $\mathcal{O}(V_{adjacentV})$ & \emph{local}\\
\hline
KillEV & $\mathcal{O}(V_{adjacentV})$ & \emph{local}\\
\hline
MakeEF & $\mathcal{O}(V_{adjacentV})$ & \emph{local}\\
\hline
KillEF & $\mathcal{O}(V_{adjacentV})$ & \emph{local}\\
\hline
MakeFkillR & $\mathcal{O}(V_{adjacentV})$ & \emph{local}\\
\hline
KillFmakeR & $\mathcal{O}(V_{adjacentV})$ & \emph{local}\\
\hline
\end{tabular}
\end{figure}
The table above shows the allowed complexity of the Euler operators on any mesh structure.

\begin{figure}[!hbf]

\centering
\textbf{Editing functions}\\
\vspace{2mm}
\begin{tabular}{| c | c | c | c |}
\hline
\textbf{Name of the operation} & \textbf{Allowed complexity} & \textbf{local/global}\\
\hline
Truncate & $\mathcal{O}(V_{adjacentV})$ & \emph{local}\\
\hline
Bevel & $\mathcal{O}(V_{adjacentV})$ & \emph{local}\\
\hline
Extrude & $\mathcal{O}(V_{adjacentV})$ & \emph{local}\\
\hline
\end{tabular}
\end{figure}
Despite to the fact the operations listed in the table above can be created using the
Euler operations a mesh may have own, faster and optimized implementation of the operation.

\begin{figure}[!hbf]

\centering
\textbf{Getters (surroundings)}\\
\vspace{2mm}
\begin{tabular}{| c | c | c | c |}
\hline
\textbf{Name of the operation} & \textbf{Allowed complexity} & \textbf{local/global}\\
\hline
Get adjacent vertices of the vertex& $\mathcal{O}(V_{adjacentV})$ & \emph{local}\\
\hline
Get adjacent edges of the vertex& $\mathcal{O}(E_{adjacentV})$ & \emph{local}\\
\hline
Get adjacent faces of the vertex& $\mathcal{O}(F_{adjacentV})$ & \emph{local}\\
\hline
Get vertices of the edge& $\mathcal{O}(V_{consistE})$ & \emph{local}\\
\hline
Get faces that the edge splits& $\mathcal{O}(F_{adjacentE})$ & \emph{local}\\
\hline
Get vertices of the face& $\mathcal{O}(V_{consistF})$ & \emph{local}\\
\hline
Get edges of the face& $\mathcal{O}(E_{consistF})$ & \emph{local}\\
\hline
Get adjacent faces of the face& $\mathcal{O}(F_{adjacentF})$ & \emph{local}\\
\hline
\end{tabular}
\end{figure}
The operations listed above does not affect the mesh; only information about the
structure is acquired. The allowed complexity does not permit scanning the entire structure.
The operations cover operations that determines whether two elements are adjacent.
E.g. the operation \emph{determine whether two vertices are adjacent} is covered by the
operation \emph{Get adjacent vertices of the vertex}.

\begin{figure}[!hbf]
\centering
\textbf{Getters (surroundings)}\\
\vspace{2mm}
\begin{tabular}{| c | c | c | c |}
\hline
\textbf{Name of the operation} & \textbf{Allowed complexity} & \textbf{local/global}\\
\hline
Get all vertices & $\mathcal{O}(V_{all})$ & \emph{global}\\
\hline
Get all edges & $\mathcal{O}(E_{all})$ & \emph{global}\\
\hline
Get all faces & $\mathcal{O}(F_{all})$ & \emph{global}\\
\hline
\end{tabular}
\end{figure}
The operation that gets all elements in the structure assumes that all elements are contained
in specific container that can be scanned in a linear complexity.