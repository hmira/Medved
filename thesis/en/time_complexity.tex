\chapter{Computational complexity}

In this chapter is described the impact on the computational complexity by choice of the mesh
representation. We will also define the limits for those we consider the operation
as \emph{effective} or \emph{ineffective}. Later in the thesis, if the operation is
marked as effective it must fulfill the defined criteria. If the mesh representation
does not contain an operation that satisfies the criteria for complexity then
the representation does not natively support the operation.

\section{Effective vs. Ineffective}

The efficiency is a relative expression so that we need to define what is effective.
If the operation does query on the local topology only then the scanning the whole structure
are considered in this case as ineffective. The acceptable complexity is the complexity
of scanning the immediate surroundings.\\

\textbf{Example:}\\
Getting all adjacent verticies of a vertex in the face-vertex representation. \ref{sec:face-vertex}\\

\textbf{Analysis:}\\
However the structure does not possess information about vertex adjacency
we can acquire the demanded information by scanning the topology of the mesh.
As we do not have any auxiliary structure the information can be acquired only by
scanning the whole structure. Thus the only way how to solve this problem
is for each face check whether the vertex is contained in the face. Then the
previous and the next vertex from the view of the demanded vertex are the ones
that forms the edges in the mesh.\\

\textbf{Conclusion:}\\
The computation complexity of the operation can be computed intuitively:
For \emph{each face} we need to check whether the demanded vertex is contained in the
face so we need to check \emph{each vertex in the face} and if the vertex is contained
we are able get an adjacent vertex in constant time.\\
\\
Thus the computational complexity of the operation
is $\mathcal{O}(F_{mesh} \times V_{face})$ in total.
However getting the adjacent vertices is the operation querying the surroundings only
the complexity of the operation on the face-vertex structure is above the effective limit.\\
\\
The limits for the operation are defined in the following paragraph.

\section{Limits}

For each operation mentioned in \ref{chap:op_al} has to be defined the specific limit
of effectiveness. Before we specify the limits, we classify the operations in two categories:

\begin{itemize}
\item \emph{Global operation} - the operation that can not be done without any information
about entire structure
\item \emph{Local operation} - the operation that involves only the specific part of the mesh
without querying the rest of elements in the structure
\end{itemize}

\begin{figure}[!hbf]

\centering
\begin{tabular}{| c | c | c | c |}
\hline
\textbf{Name of the operation} & \textbf{Complexity} & \textbf{Amortized complexity} & \textbf{local/global}\\
\hline
Truncate & $\mathcal{O}(V_{adjacent})$ & $\mathcal{O}$ & \emph{local}\\
\hline
Bevel & $\mathcal{O}(V_{adjacent})$ & $\mathcal{O}$ & \emph{local}\\
\hline
\end{tabular}

\end{figure}