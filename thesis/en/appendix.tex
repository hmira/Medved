\begin{appendices}

\chapter{Running the Implemented Examples}

The library has been tested on the OS \emph{Ubuntu 12.10} using compiler \texttt{g++ 4.7}
and has been developed using the multi-platform developement tools only.

\lstset { %
belowcaptionskip=1\baselineskip,
breaklines=true,
language=C++,
showstringspaces=false,
basicstyle=\footnotesize\ttfamily,
keywordstyle=\bfseries\color{green!40!black},
commentstyle=\itshape\color{purple!40!black},
identifierstyle=\color{blue},
stringstyle=\color{orange},
backgroundcolor=\color{black!5}
}

%\begin{lstlisting}
%for (int i=0; i<iterations; i++)
%{
%	do something
%}
%\end{lstlisting}

\section{Installation}

\begin{lstlisting}
git clone https://github.com/hmira/hmiralib.git
\end{lstlisting}

\begin{lstlisting}
mkdir build
cd build/
\end{lstlisting}

\begin{lstlisting}
cmake ..
\end{lstlisting}

\begin{lstlisting}
make [custom executable]
\end{lstlisting}

\section{Running}

By running a command

\begin{lstlisting}
./test/[custom executable] --help
\end{lstlisting}
the program prints a message that specifies the required parameters to run.

\chapter{Templating by a Custom Polygon Mesh}

To make the library work, we must build a proper traits in order to let the library recognize
the required operations.\\
\\
Depending on the demanding algorithm, we must create the traits that satisfy the concept
of the algorithm. If the concept is not fulfilled the compiler throws an error.\\

\textbf{Example:}

\begin{lstlisting}
template <typename my_mesh>
class my_mesh_implementation_traits<my_mesh>
{
public:
	typedef typename my_mesh::point point;
	typedef typename my_mesh::normal normal;

	typedef typename my_mesh::vertex_descriptor vertex_descriptor;
	typedef typename my_mesh::vertex_iterator vertex_iterator;

	typedef typename my_mesh::edge_descriptor edge_descriptor;
	typedef typename my_mesh::edge_iterator edge_iterator;

	typedef typename my_mesh::face_descriptor face_descriptor;
	typedef typename my_mesh::face_iterator face_iterator;

	typedef typename my_mesh::fv_iterator fv_iterator;
	typedef typename my_mesh::vv_iterator vv_iterator;
	typedef typename my_mesh::ve_iterator ve_iterator;
	
	inline static bool add_vertex(
				  vertex_descriptor a,
		  	  	  my_mesh& m)
	{
		m.my_add_vertex(a);
		return true;
	}		
	
	inline static bool create_face(
				  vertex_descriptor a,
				  vertex_descriptor b,
				  vertex_descriptor c,
		  	  	  my_mesh& m)
	{
		m.my_make_face(a,b,c);
		return true;
	}
}
\end{lstlisting}

From here, we have a traits that is capable of generating a cube.

\begin{lstlisting}
int main()
{
	typedef typename my_mesh_implementation_traits<my_mesh> traits;
	auto cube = generate_cube<my_mesh, traits>();
}
\end{lstlisting}

\chapter{Expansion of the Library}

\section{Creating a new Concept}

Once we have an algorithm/adapter working, we have to create a concept following the convention
provided by the library. In the concept it has to be contained the rules required for running the
algorithm/adaptor.\\

\textbf{Example:}\\
We have a method that generates a cube
\label{app:cube}
\begin{lstlisting}
template <typename TMesh, typename TMesh_traits>
bool generate_cube(TMesh& m)
{
	typedef typename TMesh_traits::Point Point; //coordinates
	typedef typename TMesh_traits::Vertex Vertex; //vertex
	typedef typename TMesh_traits::Face Face; //face type
	
	auto a = Vertex(Point(-1,-1,-1));
	auto b = Vertex(Point(1,-1,-1));
	auto c = Vertex(Point(1,-1,1));
	auto d = Vertex(Point(-1,-1,1));
	auto e = Vertex(Point(-1,1,-1));
	auto f = Vertex(Point(1,1,-1));
	auto g = Vertex(Point(1,1,1));
	auto h = Vertex(Point(-1,1,1));
	
	TMesh_traits::add_vertex(m, a);
	TMesh_traits::add_vertex(m, b);
	TMesh_traits::add_vertex(m, c);
	TMesh_traits::add_vertex(m, d);
	TMesh_traits::add_vertex(m, e);
	TMesh_traits::add_vertex(m, f);
	TMesh_traits::add_vertex(m, g);
	TMesh_traits::add_vertex(m, h);

	TMesh_traits::create_face(m, a, b, c, d);
	TMesh_traits::create_face(m, b, c, g, f);
	TMesh_traits::create_face(m, c, g, h, d);
	TMesh_traits::create_face(m, a, d, h, e);		
	TMesh_traits::create_face(m, e, f, b, a);		
	TMesh_traits::create_face(m, h, g, f, e);
	
	return true; // the cube has been generated succesfully
}
\end{lstlisting}
Thus the concept appear as follows:

\begin{lstlisting}
template <class TMesh, class TMesh_Traits>
struct GenerateCubeConcept
{
	typedef typename TMesh_traits::Point Point; //coordinates
	typedef typename TMesh_traits::Vertex Vertex; //vertex
	typedef typename TMesh_traits::Face Face; //face type

	TMesh m;
	Vertex v;
	float f;
	Point p;

	void constraints() {

		boost::function_requires<MeshConcept<TMesh, TMesh_Traits> >();

		p = Point(f, f, f);
		v = Vertex(p);
		TMesh_traits::add_vertex(m, v);
		TMesh_traits::create_face(m, v, v, v, v);
	}
};
\end{lstlisting}

After building a concept, we place a concept checker in the implementation
of the algorithm.

\begin{lstlisting}
boost::function_requires<GenerateCubeConcept<TMesh, TMesh_Traits> >();
\end{lstlisting}

\section{Creating a new Algorithm}

The algorithm should be created from the elementary operations that are commonly used in the
library. As more operations that are already used in the other algorithms
the algorithm uses than the robustness of the library is enhanced. In the other words,
user should let himself be inspired by the other algorithms so the concepts overlap as a result.
\\
\\
Inspiring from the implementation \ref{app:cube} we can implement an algorithm that generates
a tetrahedron.
\begin{lstlisting}
template <typename TMesh, typename TMesh_traits>
bool generate_cube(TMesh& m)
{
	typedef typename TMesh_traits::Point Point; //coordinates
	typedef typename TMesh_traits::Vertex Vertex; //vertex
	typedef typename TMesh_traits::Face Face; //face type
	
	auto a = Vertex(Point(0,0,0));
	auto b = Vertex(Point(0,0,1));
	auto c = Vertex(Point(0,1,0));
	auto d = Vertex(Point(1,0,0));
	
	TMesh_traits::add_vertex(m, a);
	TMesh_traits::add_vertex(m, b);
	TMesh_traits::add_vertex(m, c);
	TMesh_traits::add_vertex(m, d);

	TMesh_traits::create_face(m, a, b, c);
	TMesh_traits::create_face(m, d, a, c);
	TMesh_traits::create_face(m, d, b, a);
	TMesh_traits::create_face(m, d, c, b);
	
	return true; // the tetrahedron has been generated succesfully
}
\end{lstlisting}

From here, compared to the implementation\ref{app:cube} that generates a cube, the implementation
that generates a tetrahedron requires one method to have modified. Thus the concept will be as follows:

\begin{lstlisting}
template <class TMesh, class TMesh_Traits>
struct GenerateTetrahedronConcept
{
	typedef typename TMesh_traits::Point Point; //coordinates
	typedef typename TMesh_traits::Vertex Vertex; //vertex
	typedef typename TMesh_traits::Face Face; //face type

	TMesh m;
	Vertex v;
	float f;
	Point p;

	void constraints() {

		boost::function_requires<MeshConcept<TMesh, TMesh_Traits> >();

		p = Point(f, f, f);
		v = Vertex(p);
		TMesh_traits::add_vertex(m, v);
		TMesh_traits::create_face(m, v, v, v);
		// 3 vertices instead of 4
	}
};
\end{lstlisting}

\end{appendices}