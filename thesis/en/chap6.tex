\chapter{Adapters}

If the polygon mesh does not satisfy the designed concept of the algorithm then the compiler
raises an error. Nevertheless, in some cases, there is an option to create an additional function
that performs according the requirements for the absent operation.
In C++ standard library, adapters are used to build function objects out of ordinary
functions or class methods\cite{Simonis2000}.\\
\\
Inspired by the adapters from the C++ standard library, we create a set of adapters
for our generic algorithms library.
In this chapter is described the purpose of adapters in the implementation of generic algorithms
supported by several examples. It is primarily used in case of absence any required operation
used in algorithm.

\section{Observation}

Same as algorithms, adapters have specific requirements to be created. From this point of view,
we can state that an adapter has a capability to substitute a point of the concept with
the other ones that are required for the adapter.\\
\\
That statement is supported in following paragraphs.

\section{Get-adjacent-vertex-of-vertex adapter}

If the polygonal mesh is not representation that contains direct information about adjacency of two
vertices then the mesh can not effectively 
get the adjacent vertices from a given vertex. Obviously, no algorithm that contains query for
adjacent vertices of a vertex can not be working over such a structure.\\
\\
However, using the brute-force, we can determine which ones from vertices are in adjacency with
the queried vertex. In the other words, demanded vertices can be acquired after checking all
faces that contain the queried vertex.

\subsection{Concept of the adapter}

Assuming the structure is non-edge-based, we add an assumption that the structure is face-based
so it supports the face structure that provides information about contained vertices in correct
order. The points of the concept are as follows:

\begin{itemize}
\item mesh is required to support getting all faces of the mesh
\item mesh is required to support getting all vertices of a face in the mesh
\end{itemize}
Finally, we can conclude that the point ``\emph{mesh is required to support getting all adjacent vertices
of a vertex in the mesh}" can be substituted by these two points in a concept of any 
algorithm. In the other hand, the complexity of the operation raises from 
$\mathcal{O}(V_{vertex})$ to $\mathcal{O}(F_{mesh})$ where the $F_{mesh}$ is the number of the
faces in total and $V_{vertex}$ is the expected count of the adjacent vertices.