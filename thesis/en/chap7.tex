\chapter{Hmira library}

The library is written in C++ using the standard \emph{C++11}\cite{Naugler2013}. However the features
for metaprogramming are not yet as advanced as the ones from \emph{Boost}\cite{Abrahams2004} we
use Boost library, especially Boost Metaprogramming Library and Boost Preprocessor for
metaprogramming purposes.

\section{Designing the concept}

The important part of the generic library is the concept - the set of requirements for a template
used as parameter. After the concept is satisfied, code can be built.\\
\\
Concept consists of axioms and constraints\cite{Sutton2012}. Constraints are statically evaluable
predicates on the properties. Axioms are the requirements on the types that can not be statically
evaluated.

\section{C++11}

The standard \emph{C++11} comes with new features such as lambda functions, static assertions or other
features that help us to pre-evaluate available constants during the compile-time.\\
\\
In C++11, there is the header \texttt{\textless type\_traits\textgreater} that defines
a series of classes to obtain information during the compile-time.

\section{Boost}

Boost have a metaprogramming framework \texttt{boost::mpl}. During the compile-time, it is
capable of evaluation logic or arithmetic expressions. Combinated with \texttt{std::enable\_if}
or static assertion it can determine which function will be compiled and which not.