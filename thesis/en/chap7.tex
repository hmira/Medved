\chapter{Hmira Library}

The library is written in C++ using the standard \emph{C++11}\cite{Naugler2013}. However the features
for metaprogramming are not yet as advanced as the ones from \emph{Boost}\cite{Abrahams2004} we
use Boost library, especially Boost Metaprogramming Library\cite{bmpl}
and Boost Preprocessor Library\cite{bpl} for metaprogramming purposes.

\section{Designing the Concept}

The important part of the generic library is the concept - the set of requirements for a template
used as parameter. After the concept is satisfied, a code can be built.\\
\\
Concept consists of axioms and constraints\cite{Sutton2012}. Constraints are statically evaluable
predicates of the properties. Axioms are the requirements on the types that can not be statically
evaluated.

\section{C++11}

The standard \emph{C++11} comes with new features such as lambda functions, static assertions and other
features that help us to pre-evaluate available constants during the compile-time.\\
\\
C++11 contains the header \texttt{\textless type\_traits\textgreater} that defines
a series of classes to obtain information during the compile-time.

\section{Boost}

Boost has a metaprogramming framework \texttt{boost::mpl}\cite{bmpl}. During the compile-time, it is
capable of evaluation logical or arithmetical expressions. Combined with \texttt{std::enable\_if}
or static assertion it can determine which function will be compiled and which not.

\section{Threading Building Blocks}

The library \emph{Threading Building Blocks}\cite{Reinders2010}\cite{Pheatt2008}
is multiplatform C++ template library
for task parallelism made by Intel\textregistered. It uses C++ templates to eliminate the need to
create and manage threads. The Hmira library uses the Threading Building Blocks in algorithms
that have a section which can be done parallely; in the chapter \ref{chap:alg_decomp}, the
section which can be done paralelly is labeled with \textbf{parallel}. E.g.: the \textbf{parallel for}
does not necessarily mean that the block must be a parallel loop; it means that the block of code
provides an option being done parallelly.