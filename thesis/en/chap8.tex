\chapter{Concepts}

The example in the previous chapter\ref{chap:traits} shows that the implemented algorithm
is not be working if the class that is used as traits misses any of the methods or parameters
used in the algorithm. In order to be certain, the error is raised during the compile time.\\
\\
The set of all parameters and method required by the algorithm is called \emph{concept}.
The concept specifies what must be obtained in the traits to let the algorithm work correctly.
In this chapter is explained the purposes of specifying the concept of the algorithm and the
possible methods to let the compiler check the concept.\\
\\
Summarized from the code\ref{code:traits} of the example, we can see that the required
paramters and methods are as follows:

\begin{lstlisting}
	//parameters
	typename Traits::Point
	typename Traits::Vertex
	typename Traits::Face
	
	//methods
	Traits::Face create_face(
		Traits::Mesh& m,
		Traits::Vertex a,
		Traits::Vertex b,
		Traits::Vertex c,
		Traits::Vertex d);

	bool add_vertex(
		Traits::Mesh& m,
		Traits::Vertex a);
\end{lstlisting}
In addition, the type \texttt{Vertex} must be constructible from the type \texttt{Point}.
and the type \texttt{Point} must be constructible from the triplet of \texttt{float}.
As the syntax of C++ does not distinguish the constructor and \texttt{operator()} the
user does not have to overload the constructor; the wrapper for the class with an
operator is sufficient.