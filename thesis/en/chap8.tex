\chapter{Concepts}

The example in the chapter \ref{chap:traits} shows that the implemented algorithm
is not be working if the class that is used as traits misses any of the methods or parameters
used in the algorithm. In order to be certain, the error is raised during the compile time.\\
\\
The set of all parameters and method required by the algorithm is called \emph{concept}.
The concept specifies what must be obtained in the traits to let the algorithm work correctly.
In this chapter is explained the purposes of specifying the concept of the algorithm and the
possible methods to let the compiler check the concept.\\
\\
Summarized from the code of the example in the chapter \ref{code:traits}, we can see that the required
paramters and methods are as follows:

\begin{lstlisting}
	//parameters
	typename Traits::Point
	typename Traits::Vertex
	typename Traits::Face
	
	//methods
	Traits::Face create_face(
		Traits::Mesh& m,
		Traits::Vertex a,
		Traits::Vertex b,
		Traits::Vertex c,
		Traits::Vertex d);

	bool add_vertex(
		Traits::Mesh& m,
		Traits::Vertex a);
\end{lstlisting}
In addition, the type \texttt{Vertex} must be constructible from the type \texttt{Point}.
and the type \texttt{Point} must be constructible from the triplet of \texttt{float}.
From this point there are two approaches to create a class that fullfills the requirement
for the constructor assuming the class does not support it by default.
The first, to create an inherited class with modified constructor. The second,
to build a templated wrapper that uses a class as a parameter.
\label{code:constr1}
\begin{lstlisting}
	//building a vertex type
	//using inheriting class
	class Vertex1 : public MyVertex
	{
	public:
		Vertex1( MyPoint& p )
		{
			this->coordinates = p;
		}
	};
	//building a vertex type
	//using a wrapper class
	template <typename TVertex, typename TPoint>
	class Vertex2 : public TVertex
	{
	public:
		Vertex2( TPoint& p )
		{
			this->coordinates = p;
		}
	}
\end{lstlisting}
The code implements an example with an assumption that the class \texttt{MyVertex}
contains a member \texttt{coordinates}.

\section{Usage}

The user can check the concept by the placing the following code before the calling of the function.
\begin{lstlisting}
boost::function_requires<GenerateCubeConcept<Mesh, Traits >();
\end{lstlisting}
If the concept is not satisfied, the error code will apear as follows:

\lstset {language=bash,
keywordstyle=\color{black},
basicstyle=\ttfamily,
keywordstyle=[2]{\color{black}},
literate={\$}{{\textcolor{green}{\$}}}1 
         {:}{{\textcolor{black}{:}}}1
         {~}{{\textcolor{red}{\textasciitilde}}}1,
}

\begin{lstlisting}
$ /usr/include/boost/concept/detail/has_constraints.hpp:32:14:   required by substitution of template<class Model> boost::concepts::detail::yes boost::concepts::detail::has_constraints_(Model*, boost::concepts::detail::wrap_constraints<Model, (& Model:: constraints)>*) [with Model = VertexAdjacencyConcept<Mesh, Traits > >]
\end{lstlisting}
If the checking is not processed, the compilator still throws an error. The benefit of the error
message above is that the compiler always throws such an error if the concept is not satisfied.
If the concept is not checked, the compiler may produce a misleading or confusing error messages
that are more demanding to solve.

\lstset { %
belowcaptionskip=1\baselineskip,
breaklines=true,
language=C++,
showstringspaces=false,
basicstyle=\footnotesize\ttfamily,
keywordstyle=\bfseries\color{green!40!black},
commentstyle=\itshape\color{purple!40!black},
identifierstyle=\color{blue},
stringstyle=\color{orange},
backgroundcolor=\color{black!5}
}