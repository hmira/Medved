\chapter{Libraries}

What we did not yet mentioned is the implemented data structures that fulfills the described rules.
In this chapter there are introduced several libraries that are comonnly used. In following paragraphs
is shown the corelation with the structures shown in the previous chapter.

\section{OpenMesh}

\emph{OpenMesh} is an open-source generic data structure for representing and manipulating polygonal
meshes. It is developed at the Computer Graphics Group, RWTH Aachen.
\\
\\
Restricting to the meshes introduced in chapter 2, OpenMesh is considered as a half-edge structure.
It is formed by kernel that set proper attributes such as triangular restriction or allowing to remove
vertices from mesh.
\\
To fully specify a mesh, several parameters can be given:\\

\textbf{Face type}: Specifies whether to use a general polygonal mesh or the triangle mesh.\\

\textbf{Kernel}: Stores the element of the mesh internally. User chooses from available
kernels according to expected usage. For example the decimation algorithm requires
efficient deletion/insertion thus the proper kernel for this case is kernel based on
linked list.\\

\textbf{Traits}: Traits is the class that enhances the mesh functionality such as
vertices removal or adding various attributes to elements.