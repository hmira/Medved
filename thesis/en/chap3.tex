\chapter{Libraries}

We did not yet mention the implemented data structures that fulfill the described rules.
This chapter introduces several libraries that are comonnly used. The following paragraphs show 
the corelation with the structures shown in the chapter \ref{chap:op_al}.

\section{OpenMesh}

\emph{OpenMesh}\cite{OMesh}\cite{Botsch2002} is an open-source generic data structure for representing
and manipulating polygonal meshes. It has been developed at the Computer Graphics Group, RWTH Aachen.
\\
\\
Restricting to the meshes introduced in the section \ref{sec:polyg_mesh},
OpenMesh is considered as a half-edge structure.
It is formed by the kernel that set proper attributes such as triangular restriction or allowing to remove
vertices from mesh.
\\
To fully specify a mesh, several parameters can be given:\\

\textbf{Face type}: Specifies whether to use a general polygonal mesh or a triangle mesh.\\

\textbf{Kernel}: Stores the element of the mesh internally. User chooses from the available
kernels according to expected usage. For example, the decimation algorithm requires
efficient deletion/insertion, thus the proper kernel for this case is the kernel based on
linked list.\\

\textbf{Traits}: Traits is the class that enhances the mesh functionality such as
vertices removal or adding various attributes to elements.

\section{Trimesh}

Trimesh is the library designed to read, write, and manipulate with the triangle meshes\cite{trimesh}.\\
\\
Compared to the \emph{OpenMesh} library, Trimesh library emphasizes the efficiency
and easy of use rather than the sophisticated design. The representation of the mesh
is a modified face-vertex structure(see the description in the section \ref{sec:face-vertex}).
The modification consist in the addition of 3 connectivity structures:
\begin{itemize}
\item \texttt{neighbors} - for each vertex, all adjacent vertices
\item \texttt{adjacentfaces} - for each vertex, all adjacent faces
\item \texttt{across\_edge} - for each face, the three faces attached to its edge. Since the
faces are restricted to triangles, a given face is allowed to have only the triplet of adjacent faces.
\end{itemize}