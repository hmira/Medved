\chapter{Adapters}

The chapter \ref{chap:adapter_analysis} explains the purpose of the adapters.
The adapter is worth using only if the conversion of the used mesh representation
is more demanding than the usage of the adapter.\\
\\Adapters generally can not
be treated as standard implementation of the function that an adapter replaces;
the computational complexity of the adapter increases the total complexity of the
algorithm so that the resulting behavior is considered as \emph{inefficient}
(see the section \ref{sec:eff_vs_ineff}).

%The user has to consider the
%efficiency (see the section \ref{sec:eff_vs_ineff}) that is acceptable for the use.

\section{Usage}

The adapters are processing the demanded operation by the \emph{brute-force}. When
the user is building the traits(see the chapter \ref{chap:traits}) for the algorithm,
he puts a calling of the adapter inside the function placed to traits instead of
calling the function supported by a mesh structure.\\

\textbf{Example:}\\
Let us have an algorithm \texttt{X} that requires iterating through all adjacent vertices
from a given vertex. We have a \emph{face-vertex} polygonal mesh(see the section \ref{sec:face-vertex})
that does not support the \emph{efficient}(see the section \ref{sec:eff_vs_ineff}) operation
capable of getting the all adjacent vertices.\\
\\
\emph{How do we build a traits for the algorithm} \texttt{X}\emph{?}\\
\\
Let us assume that the type representing the mesh is named \texttt{X\_fv\_mesh}, and
the required operation of the algorithm that has to be contained in traits is as follows:

\begin{lstlisting}
static std::pair<vv_iterator,vv_iterator>
get_adjacent_vertices(
	const X_fv_mesh& m,
	const vertex_descriptor v);
\end{lstlisting}

\textbf{Observation:}\\
From the declaration of the function above, it is obvious that except the \texttt{X\_fv\_mesh},
two additional types has to be contained in the traits.

\begin{lstlisting}
typedef vertex_descriptor X_fv_mesh&::Vertex; //supported by the structure
typedef vv_iterator ??? //not supported by the structure
\end{lstlisting}

Thus the question arises ``How do we define the \texttt{vv\_iterator}" and what do we
place into the implementation of the function \texttt{get\_adjacent\_vertices()}?\\

\textbf{Solution:}\\
We use an adapter \texttt{hmira::adapters::vv\_adapter::vv\_iterator\_adapter} which allow us to
call a function that returns the demanded pair of the iterators and provides a type
that was required for the function; in case of this example, \texttt{vv\_iterator}.\\
\\
The implementation of the traits will appear as follows:
\nopagebreak

\begin{lstlisting}
#include <hmira/adapters/vv_adapter.hpp>

class X_fv_mesh_traits
{
//the following block of the code has to be
//added to the traits
//...

	typedef typename hmira::adapters::vv_adapter<
		X_fv_mesh,
		X_fv_mesh_traits
	>::vv_iterator vv_iterator;

	static inline
	std::pair<vv_iterator, vv_iterator>
	get_adjacent_vertices(
		const X_fv_mesh& m,
		vertex_descriptor v)
	{
		return hmira::adapters::vv_adapter<
				X_fv_mesh,
				X_fv_mesh_traits
			>::vv_iterator_adapter(m, v);
	}

//...
};
\end{lstlisting}
The class \texttt{X\_fv\_mesh} does not support the operation that returns the iterators
that are able to iterate through the adjacent vertices of a given vertex. Instead of calling
the implemented function, the function refers to an adapter. In this case \texttt{vv\_iterator\_adapter}.







